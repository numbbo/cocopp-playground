${\bf f_{52}}$ & 120&(212) & 125&(218) & 133&(220) & 139&(220) & 144&(220) & 157&(222) & 163&(222) & 15 & /15\\